\documentclass{article}

% Language setting
% Replace `english' with e.g. `spanish' to change the document language
\usepackage[english]{babel}

% Set page size and margins
% Replace `letterpaper' with `a4paper' for UK/EU standard size
\usepackage[letterpaper,top=2cm,bottom=2cm,left=3cm,right=3cm,marginparwidth=1.75cm]{geometry}

% Useful packages
\usepackage{amsmath}
\usepackage{graphicx}
\usepackage[colorlinks=true, allcolors=blue]{hyperref}

\title{Introducción a Packet Tracer}
\author{Juan Cruz Villena}


\begin{document}
\maketitle


\section{Packet Tracer}

Permite crear simulaciones de entornos de redes.
Se pueden agregar dispositivos y cables para conectar los dispositivos entre sí.
Cada elemento se puede seleccionar, inspeccionar o eliminar.
Packet Tracer cuenta con varios elementos, en el cursado estaremos viendo dispositivos finales, dispositivos de red y conexiones.
Se pueden enviar paquetes entre dispositivos conectados.

\subsection{Dispositivos Finales}
Son los puntos finales de las conexiones.
\begin{enumerate}
\item PC Contiene una serie de modulos que se le pueden agregar, su configuración que se puede editar, una serie de aplicaiones de escritorio, un entorno de programacion y sus atributos.
\item Laptop. Similar a la PC, unicamente cuenta con diferentes modulos y diferente estructura fisica.
\item Servidor. Similar a PC, con diferente estructura fisica y permite brindar servicios de red.
\end{enumerate}

\subsection{Dispositivos de Red}
Permiten conectar 2 o más dispositivos al funcionar como intermediarios.
\begin{enumerate}
\item Routers. Tienen estructura fisica, configuracion, una interfaz de linea de comandos y atributos
\item Switchers. Similar a Routers pero menos modulos y más puertos.
\item Hubs. Estructura fisica, configuracion y atributos.
\end{enumerate}

\subsection{Cableado}

\begin{enumerate}
\item Copper Straight-Through: Cable directo para conectar pc a hub o switch.
\item Copper Cross-Over: Se utiliza para conectar switchs o hubs entre sí.
\end{enumerate}

\subsection{Conexión de 2 Computadoras}
2 PCs se pueden conectar mediante un cable cruzado básico, aunque solo permiten la conexión de 2 dispositivos. Para conectar otro dispositivo se necesitaria el uso de un hub o switch.

\cite{juancruz}
\bibliographystyle{alpha}
\bibliography{sample}
\href{https://github.com/Juancruz2003/TP-Packet-Tracer}{https://github.com/Juancruz2003/TP-Packet-Tracer}

\end{document}